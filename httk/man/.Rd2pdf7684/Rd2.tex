\documentclass[a4paper]{book}
\usepackage[times,inconsolata,hyper]{Rd}
\usepackage{makeidx}
\usepackage[utf8,latin1]{inputenc}
% \usepackage{graphicx} % @USE GRAPHICX@
\makeindex{}
\begin{document}
\chapter*{}
\begin{center}
{\textbf{\huge \R{} documentation}} \par\bigskip{{\Large of \file{armitage\_eval2.Rd}}}
\par\bigskip{\large \today}
\end{center}
\inputencoding{utf8}
\HeaderA{armitage\_eval}{Evaluate the updated Armitage model}{armitage.Rul.eval}
%
\begin{Description}\relax
Evaluate the Armitage model for chemical distributon in vitro. Takes input
as data table or vectors of values. Outputs a data table. Updates over
the model published in Armitage et al. 2014 include binding to plastic walls
and lipid and protein compartments in cells.
\end{Description}
%
\begin{Usage}
\begin{verbatim}
armitage_eval(
  casrn.vector = NA_character_,
  nomconc.vector = 1,
  this.well_number = 384,
  this.FBSf = NA_real_,
  tcdata = NA,
  this.sarea = NA_real_,
  this.v_total = NA_real_,
  this.v_working = NA_real_,
  this.cell_yield = NA_real_,
  this.Tsys = 37,
  this.Tref = 298.15,
  this.option.kbsa2 = F,
  this.option.swat2 = F,
  this.pseudooct = 0.01,
  this.memblip = 0.04,
  this.nlom = 0.2,
  this.P_nlom = 0.035,
  this.P_dom = 0.05,
  this.P_cells = 1,
  this.csalt = 0.15,
  this.celldensity = 1,
  this.cellmass = 3,
  this.f_oc = 1
)
\end{verbatim}
\end{Usage}
%
\begin{Arguments}
\begin{ldescription}
\item[\code{casrn.vector}] For vector or single value, CAS number

\item[\code{nomconc.vector}] For vector or single value, micromolar nominal 
concentration (e.g. AC50 value)

\item[\code{this.well\_number}] For single value, plate format default is 384, used
if is.na(tcdata)==T

\item[\code{this.FBSf}] Fraction fetal bovine serum, must be entered by user.

\item[\code{tcdata}] A data.table with casrn, nomconc, MP, gkow, gkaw, gswat, sarea,
v\_total, v\_working. Otherwise supply single values to this.params.

\item[\code{this.sarea}] Surface area per well (m\textasciicircum{}2)

\item[\code{this.v\_total}] Total volume per well (m\textasciicircum{}3)

\item[\code{this.v\_working}] Working volume per well (m\textasciicircum{}3)

\item[\code{this.cell\_yield}] Number of cells per well

\item[\code{this.Tsys}] System temperature (degrees C)

\item[\code{this.Tref}] Reference temperature (degrees K)

\item[\code{this.option.kbsa2}] Use alternative bovine-serum-albumin partitioning
model

\item[\code{this.option.swat2}] Use alternative water solubility correction

\item[\code{this.pseudooct}] Pseudo-octanol cell storage lipid content

\item[\code{this.memblip}] Membrane lipid content of cells

\item[\code{this.nlom}] Structural protein conent of cells

\item[\code{this.P\_nlom}] Proportionality constant to octanol structural protein

\item[\code{this.P\_dom}] Proportionality constant to dissolve organic material

\item[\code{this.P\_cells}] Proportionality constant to octanol storage lipid

\item[\code{this.csalt}] Ionic strength of buffer, mol/L

\item[\code{this.celldensity}] Cell density kg/L, g/mL

\item[\code{this.cellmass}] Mass per cell, ng/cell

\item[\code{this.f\_oc}] 1, everything assumed to be like proteins
\end{ldescription}
\end{Arguments}
%
\begin{Value}

\Tabular{lll}{
\strong{Column} & \strong{Description} & \strong{units} \\{}
casrn & Chemical Abstracts Service Registry Number & \\{}
nomconc & Nominal Concentration & mol/L \\{}       
well\_number & Number of wells in plate & unitless \\{}   
sarea & Surface area of well & m\textasciicircum{}2 \\{}         
v\_total & Total volume of well & m\textasciicircum{}3 \\{}       
v\_working & Filled volume of well & m\textasciicircum{}3 \\{}     
cell\_yield & Number of cells & cells \\{}    
gkow & log10 octanol-water partition coefficient & log10 \\{}          
logHenry & log10 Henry's law constant '& log10 atm-m3/mol \\{}      
gswat & log10 Water solubility & log10 mol/L \\{}         
MP & Melting Point & degrees Celsius \\{}           
MW & Molecular Weight & g/mol \\{}            
gkaw & air-water partition coefficient & (mol/m3)/(mol/m3) \\{}  
dsm & & \\{}           
duow & & \\{}          
duaw & & \\{}          
dumw & & \\{}          
gkmw & & \\{}          
gkcw & & \\{}          
gkbsa & & \\{}         
gkpl & & \\{}          
ksalt & & \\{}        
Tsys & & \\{}          
Tref & & \\{}          
option.kbsa2 & & \\{}  
option.swat2 & & \\{}  
FBSf & & \\{}          
pseudooct & & \\{}     
memblip & & \\{}       
nlom & & \\{}          
P\_nlom & & \\{}   
P\_dom & dissolved organic matter b water partition coefficient & Dimesnsionless\\{}         
P\_cells & & \\{}      
csalt & & \\{}         
celldensity & & \\{}   
cellmass & & \\{}      
f\_oc & & \\{}          
cellwat & & \\{}       
Tcor & & \\{}          
Vm & Volume of media & L \\{}            
}
\end{Value}
%
\begin{Author}\relax
Greg Honda
\end{Author}
%
\begin{References}\relax
Armitage, J. M.; Wania, F.; Arnot, J. A. Environ. Sci. Technol. 
2014, 48, 9770-9779. https://doi.org/10.1021/es501955g

Honda et al. PloS one 14.5 (2019): e0217564. https://doi.org/10.1371/journal.pone.0217564
\end{References}
%
\begin{Examples}
\begin{ExampleCode}

library(httk)

# Check to see if we have info on the chemical:
"80-05-7" %in% get_cheminfo()

#We do:
temp <- armitage_eval(casrn.vector = c("80-05-7", "81-81-2"), this.FBSf = 0.1,
this.well_number = 384, nomconc = 10)
print(temp$cfree.invitro)

# Check to see if we have info on the chemical:
"793-24-8" %in% get_cheminfo()

# Since we don't look up phys-chem from dashboard:
cheminfo <- data.frame(
  Compound="6-PPD",
  CASRN="793-24-8",
  DTXSID="DTXSID9025114",
  logP=4.27, 
  logHenry=log10(7.69e-8),
  logWSol=log10(1.58e-4),
  MP=	99.4,
  MW=268.404
  )
  
# Add the information to HTTK's database:
chem.physical_and_invitro.data <- add_chemtable(
 cheminfo,
 current.table=chem.physical_and_invitro.data,
 data.list=list(
 Compound="Compound",
 CAS="CASRN",
  DTXSID="DTXSID",
  MW="MW",
  logP="logP",
  logHenry="logHenry",
  logWSol="logWSol",
  MP="MP"),
  species="Human",
  reference="CompTox Dashboard 31921")

# Run the Armitage et al. (2014) model:
out <- armitage_eval(
  casrn.vector = "793-24-8", 
  this.FBSf = 0.1,
  this.well_number = 384, 
  nomconc = 10)
  
print(out)

\end{ExampleCode}
\end{Examples}
\printindex{}
\end{document}
